\section{Scaling Raft With Dynamic Timeouts}

Like all other algorithms of its class, Raft has similar problems scaling. Though it has a distrinctly unique set of problems. It should be noted that we can consider Raft generally \textit{synchronous}. That is, all nodes must reach a \textit{univalent} state, before handling the replication of the next update to the log. Raft's use of a \textit{leader}, requires all changes to the network state be processed through a single node, thereby essentially creating a synchronous cluster where in which \textit{followers} have very little power to influence the cluster's state when there is a healthy \textit{leader}. So when we discuss scaling in terms of a leader based methods we have to examine different properities of the algorithm.

Dinghy works as an embedded algorithm within Raft. It runs side by side the core Raft \textit{follower} routines, in order to actively maintain historical statistics on a node. In its most base form Dinghy attempts to get a sense of the environment that a given server is running in and adjust some of its previously set, static parameters, and adjust them to a more appropriate level for the situation. The Lifeguard system works in a similar fashion, but with differing goals, and approaches, using a step function to help stop false positives, whereas Dinghy looks at historical averages \cite{Lifeguard}.