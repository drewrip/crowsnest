\section{Conclusion}

It becomes more and more difficult for pure Raft to achieve distributed agreement with more participants in the process. More messages have to be exchanged in order to perform the same operation, while a leader can only process communication from so many nodes at a time. We presented an algorithm that can be used in order to optimize the \textit{heartbeat timeout} and \textit{interval} of a node in order to allow for greater message throughput, and a quicker detection of faulty \textit{leaders}.

The Dinghy algorithm uses the average of the elapsed message times, in order to gain a better idea of network latency. Using this measure, timeouts can be recalculated to allow for a greater efficiency in handling a growing number of messages. So in use with a larger cluster, Dinghy can be utilized to provide better horizontal scaling. In the future we hope to be able to combine this with other methods to achieve a well rounded approach to the scaling problem. Ideally in the near future we can make it more plausible to utilize an increasing number of nodes in the replication of data.