\section{Scaling Distributed Systems}

In practical applications, clusters of varying sizes are required. In some cases many nodes will be used, each replicating a small piece of data many times over. While in others, few nodes wil are used and larger chunks of data are replicated. Though in implementation, there are drawbacks to having a cluster with many nodes. As you continue to increase node number, various factors can lower a networks required time to reach consensus. Raft solves the consensus problem algorithmically, but let's, for example, take a look at a real world example where scaling comes into play.

We can imagine a large party of friends trying to decide where they want to go to eat for dinner. In this scenario in order to make an effective, and satisfying decision as to where the group should dine, each member of the group must be consulted. So the time in which it takes the entire group to come to an agreement increases as the size of the group increases.

In this large party, many more people will have to be consulted, and more options will have to be weighed before a choice is made. Though, compare this to just a few friends, who would be able to reach mutual agreement much faster, as they have less to consider, and fewer people that need to be taken into account before reaching a decision.

This principle is clearly demonstrated in distributed systems \cite{NeumanScaling}. Adding more nodes to a cluster makes it more difficult for the network to handle faults and replicate its state. Demonstrated in Raft, the more \textit{followers} in a system, the more heartbeats that have to be sent out by the \textit{leader}, processed, responded to, and then confirmed \cite{OngaroRaft}.