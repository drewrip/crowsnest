\section{Testing Consensus}

	The big draw of Raft is its practicallity. It has allowed many the opportunity to implement distributed consensus, as it was designed to be understandable. So with this is mind, we maintain a similar dogma when it comes to testing the algorithms scalability. We wanted to design a test that would stress test the networks throughput given a certain size, given that Raft is overwhelmingly used in creating fault tolerant databases \cite{etcd, CockroachDB, TiKV, RethinkDB}.

	With these considerations, we first decided the most common use case for this method of distributed consensus. The official Raft website contains a useful list of implementations \cite{RaftSite}. Though none of these would illustrate the problem that needs to be solved. One of the fundemental benefits of a consensus algorithm is their fault tolerance. So, we elected to simplify the problem down to its core component: how quickly could a cluster recover from a downed leader. We developed a simulation that would test how long a cluster of some size would recover. With this simulation written we then pitted our proposed algorithm Dinghy, against the vanilla version of Raft.

	The code for this project is running, is called AvailSim. It is a variation on a test developed for the Dr. Ongaro's original Raft dissertation, written in Go \cite{AvailSim}. We use this specific procedure in the testing of both our proposed algorithm, Dinghy, and the benchmarks of pure Raft. We also used a variation of testing and prototyping scripts in the process of development \cite{Dinghy}.